\documentclass[a4paper,12pt]{article}

% Pakete für zusätzliche Funktionen
\usepackage[utf8]{inputenc}   % UTF-8 Zeichencodierung
\usepackage[T1]{fontenc}     % Korrekte Silbentrennung und Umlaute
\usepackage[ngerman]{babel}  % Deutsche Spracheinstellungen
\usepackage{amsmath, amssymb} % Mathematische Symbole
\usepackage{graphicx}        % Einfügen von Bildern
\usepackage{hyperref}        % Hyperlinks
\usepackage{geometry}        % Seitenränder
\geometry{a4paper, margin=2.5cm}
\usepackage{listings}        % Code-Listings
\usepackage{csquotes}        % Zitate
\usepackage{xurl}
% Einstellungen für Hyperlinks
\hypersetup{
    colorlinks=true,
    linkcolor=blue,
    citecolor=blue,
    filecolor=magenta,
    urlcolor=cyan,
    pdftitle={Webtechnologies and Applications (WTA)},
    pdfpagemode=FullScreen,
}

% Titel und Autoreninformationen
\title{\textbf{Webtechnologies and Applications (WTA)}}
\author{Kürsat Darcan | MFWS422A}
\date{Abgabedatum: \today}

\begin{document}

% Deckblatt
\maketitle
\thispagestyle{empty}
\vspace{2cm}
\begin{center}
    \includegraphics[width=0.3\textwidth]{FHDW_Logo_RGB-01.svg.png} % Ersetzen Sie "example-image" durch Ihr Logo/Bild
    \\
    \vspace{1cm}
    \textbf{Studiengang: Wirtschaftsinformatik}\\
    \textbf{Fachhochschule der Wirtschaft (FHDW)}
\end{center}
\newpage

% Inhaltsverzeichnis
\renewcommand{\thepage}{\roman{page}} % Seitenzahlen in römischen Ziffern für Verzeichnisse
\tableofcontents
\newpage

% Bei Bedarf Abbildungsverzeichnis
%\listoffigures
%\addcontentsline{toc}{section}{Abbildungsverzeichnis}
%\newpage

% Bei Bedarf Tabellenverzeichnis
%\listoftables
%\addcontentsline{toc}{section}{Tabellenverzeichnis}
%\newpage

% Abkürzungsverzeichnis
%\section*{Abkürzungsverzeichnis}
%\addcontentsline{toc}{section}{Abkürzungsverzeichnis}
%\begin{description}
%    \item[HTML] Hypertext Markup Language
%\end{description}
%\newpage

% Hauptteil mit arabischen Seitenzahlen
\renewcommand{\thepage}{\arabic{page}} % Seitenzahlen in arabischen Ziffern für den Hauptteil
\setcounter{page}{1}


% Einleitung
\section{Einleitung}
\subsection{Hintergrund und Relevanz des Themas}
Mit dem Fortschritt der künstlichen Intelligenz (KI) entwickeln sich auch Deepfake-Technologien stetig weiter, 
wodurch die öffentliche Aufmerksamkeit für diese Thematik zunimmt. 
Diese Technologien ermöglichen die Manipulation digitaler Medieninhalte wie Videos, 
Bilder und Audiodateien, sodass diese täuschend echt wirken, 
obwohl sie nie in dieser Form erstellt wurden. 
Deepfakes sind jedoch keine eigenständige Technologie, 
sondern basieren auf einer Kombination verschiedener KI-Methoden. \cite{BVDW2024}\\

\subsection{Problemstellung und Herausfoderung}
Die vielseitigen Einsatzmöglichkeiten von Deepfakes reichen von kreativen Anwendungsbereichen, 
wie der Erstellung von Unterhaltung, bis hin zu manipulativen oder schädlichen Absichten. 
So können Deepfakes gezielt eingesetzt werden, um die öffentliche Meinung zu beeinflussen, 
Personen oder Unternehmen zu schädigen oder Desinformation zu verbreiten. \cite{CounterExtremism2020}\\

\subsection{Ziel der Ausarbeitung und Aufbau}
Diese Ausarbeitung untersucht mögliche Gegenmaßnahmen zur Erkennung und Bekämpfung von Deepfake-Technologien. 
Dabei wird auch die ethische Seite betrachtet, 
insbesondere die Frage, welche Verantwortung Entwickler und Nutzer dieser Technologie tragen. 
Abschließend wird ein Ausblick auf die zukünftige Entwicklung der Technologie und deren mögliche Auswirkungen auf die Gesellschaft gegeben.
\newpage

% Gegenmaßnahmen gegen Deepfakes
\section{Gegenmaßnahmen gegen Deepfakes}
Bevor auf die Gegenmaßnahmen eingegangen wird,
ist es wichtig zu verstehen, wie genau Deepfakes erstellt beziehungsweise generiert wird.
Durch das voranschreiten der künstlichen Intelligenz und im maschin learning entstand
das Generative Adversarial Networks(GANs) durch Ian Goodfellow im Jahre 2014.
Die GAN besteht aus zwei neuronalen Netzwerken, die gegeneinander trainieren.
Diese beiden Netzwerke werden Generator und Diskriminator bezeichnet.\\
Beim Generator werden Deepfakes erstellt beziehungsweise neue Images erzeugt,
die soweit wie möglich realistisch aussehen.\\ 
Im Diskriminator werden die Trainingsdaten mit den generierten Images vermischt.
Sodass der Diskriminator dann entscheidet, ob es ein Orginal oder generiertes(Fake) Image ist.\\
Durch diese zusammenarbeit versuchen sich die Netzwerke zuverbessern, indem der Generator versucht realistische Images zuerstellen und der Diskriminator die 
generierten Images zu erkennen.\cite{BVDW2024}

\subsection{KI-Erkennungsmechanismen}
Nachdem die Funktionsweise von Deepfakes erklärt wurde, sollte klar sein,
dass die Erkennung von Deepfakes sich kompliziert gestaltet.
Auch beim Erkennungsmechanismen werdern unterschiedliche technologien, algorithmische und menschliche Ansätze kombiniert.
Um Deepfakes zu erkennen, werden aktuell Methoden verwendet, die auf spezifische Mustern in der Bild oder Tonverarbeitung gesucht.
Es werden Unregelmäßigkeiten in den Bildern oder Videos gesucht, die für das menschliche Auge schwer zu erknnen,
aber für das Algorithmus leichter zu erkennen sind.\\
Ein weiteres Ansatz ist die Verwendung von Convolutional Neural Networks(CNNs)
die auf subtile Anomalien trainiert werden, wie auf Gesichtbewegungen oder Sprachmodulationen.
Auch werden die Blinzelmustern analysiert, die bei Deepfakes oftmals fehlen oder unnatürlich erscheinen.\cite{BVDW2024}

\subsection{Deklarierung von KI-generierten Medien}

\cite{TheVerge2024a}\cite{TheVerge2024b}\cite{Meta2024}

\subsection{Prävention und Bildung}
\subsubsection{Öffentliche Aufklärung}
\newpage

% Ethik und Deepfakes
\section{moralische Herausforderungen}
\subsection{Verantwortung der Entwickler}
\subsubsection{Innovation vs. Missbrauch}
\subsubsection{Regulierung von Open-Source-Projekten}
\subsection{Gesellschaftliche Folgen}
\subsubsection{Manipulation der öffentlichen Meinung}
\subsubsection{Persönliche Schäden}
\newpage

% Zukunftsperspektiven
\section{Zukunftsperspektiven}s
\subsection{Technologische Entwicklungen}
\subsection{Gesellschaftliche Anpassungen}
\newpage

%Fazit
\section{Fazit}



% Literaturverzeichnis
\newpage
\addcontentsline{toc}{section}{Literaturverzeichnis}
\section*{Literaturverzeichnis}
\begin{thebibliography}{99}
    \bibitem{Arxiv2020} Nguyen, H. H., Yamagishi, J., \& Echizen, I. (2020). Capsule-Forensics: Using Capsule Networks to Detect Forged Images and Videos. 
    Verfügbar unter: \url{https://arxiv.org/pdf/2003.08685} (zuletzt aufgerufen am 23.01.2025). 
    \bibitem{Basecamp2024} Basecamp Digital. (2024). KI verstehen: Update zu Kennzeichnungspflichten und zum neuen ChatGPT. 
    Verfügbar unter: \url{https://www.basecamp.digital/ki-verstehen-update-zu-kennzeichnungspflichten-und-zum-neuen-chatgpt/} (zuletzt aufgerufen am 23.01.2025).
    \bibitem{BPB2024} Bundeszentrale für politische Bildung (BPB). (2024). Regulierung von Deepfakes. 
    Verfügbar unter: \url{https://www.bpb.de/lernen/bewegtbild-und-politische-bildung/556822/regulierung-von-deepfakes/} (zuletzt aufgerufen am 23.01.2025). 
    \bibitem{BPBDeepfake2024} Bundeszentrale für politische Bildung (BPB). (2024). Technische Ansätze zur Deepfake-Erkennung und -Prävention. 
    Verfügbar unter: \url{https://www.bpb.de/lernen/bewegtbild-und-politische-bildung/556855/technische-ansaetze-zur-deepfake-erkennung-und-praevention/} (zuletzt aufgerufen am 23.01.2025).
    \bibitem{BVDW2024} Bundesverband Digitale Wirtschaft (BVDW). (2024). Deepfakes – Eine juristische Einordnung. 
    Verfügbar unter: \url{https://www.bvdw.org/wp-content/uploads/2024/07/2024_BVDW_Deepfakes.pdf} (zuletzt aufgerufen am 22.01.2025). Kopie aufbewahrt als.
    \bibitem{Bundesdruckerei2024} Bundesdruckerei. (2024). Desinformation und Fake-ID. 
    Verfügbar unter: \url{https://www.bundesdruckerei.de/de/innovation-hub/desinformation-und-fake-id} (zuletzt aufgerufen am 23.01.2025). 
    \bibitem{CounterExtremism2020} Counter Extremism Project. (2020). Deepfakes – Bedrohungspotenziale und Gegenmaßnahmen.
    Verfügbar unter: \url{https://www.counterextremism.com/sites/default/files/200806_Deep_Fakes_DE_WEB_DS.pdf} (zuletzt aufgerufen am 22.01.2025). Kopie aufbewahrt als.
    \bibitem{Fraunhofer2024} Fraunhofer ISI. (2024). Chancen und Risiken von Deepfakes für Politik, Wirtschaft und Gesellschaft. 
    Verfügbar unter: \url{https://www.isi.fraunhofer.de/de/presse/2024/presseinfo-17-deepfakes-chancen-risiken-Politik-Wirtschaft-Gesellschaft.html} (zuletzt aufgerufen am 23.01.2025). 
    \bibitem{Kaur2024} Kaur, A., Noori Hoshyar, A., Saikrishna, V., Firmin, S., \& Xia, F. (2024). Deepfake video detection: challenges and opportunities. Verfügbar unter: 
    \url{https://link.springer.com/content/pdf/10.1007/s10462-024-10810-6.pdf} (zuletzt aufgerufen am 23.01.2025).
    \bibitem{Meta2024} Meta. (2024). How AI-generated content is identified and labeled on Meta. 
    Verfügbar unter: \url{https://www.meta.com/de-de/help/artificial-intelligence/how-ai-generated-content-is-identified-and-labeled-on-meta/} (zuletzt aufgerufen am 23.01.2025).
    \bibitem{Springer2024} Dittmann, J., \& Vielhauer, C. (2024). Technische Ansätze zur Deepfake-Erkennung. In: Künstliche Intelligenz und digitale Ethik. Springer. 
    Verfügbar unter: \url{https://link.springer.com/chapter/10.1007/978-3-662-65964-9_10} (zuletzt aufgerufen am 23.01.2025).
    \bibitem{Tibor2024} Tibor.net. (2024). Instagram markiert Posts mit „KI-generiert“ – was bedeutet das? 
    Verfügbar unter: \url{https://tibor.net/digital/kuenstliche-intelligenz/instagram-markiert-posts-mit-mit-ki-generiert-was-bedeutet-das/} (zuletzt aufgerufen am 23.01.2025).
    \bibitem{TheVerge2024a} The Verge. (2024). YouTube introduces C2PA standard for captured camera labels and content credentials. 
    Verfügbar unter: \url{https://www.theverge.com/2024/10/15/24271083/youtube-c2pa-captured-camera-label-content-credentials} (zuletzt aufgerufen am 23.01.2025).
    \bibitem{TheVerge2024b} The Verge. (2024). Facebook and Instagram update AI label for edited content. 
    Verfügbar unter: \url{https://www.theverge.com/2024/9/12/24242998/facebook-instagram-ai-label-update-edited-content} (zuletzt aufgerufen am 23.01.2025).
    \bibitem{BSI2025} Bundesamt für Sicherheit in der Informationstechnik (BSI). (2025). Deepfakes – Bedrohungspotenziale und Gegenmaßnahmen. 
    Verfügbar unter: \url{https://www.bsi.bund.de/DE/Themen/Unternehmen-und-Organisationen/Informationen-und-Empfehlungen/Kuenstliche-Intelligenz/Deepfakes/deepfakes_node.html} (zuletzt aufgerufen am 21.01.2025).
\end{thebibliography}


\newpage
% Anhang
\appendix
\section{Anhang}
Hier können zusätzliche Informationen, wie Code-Beispiele oder ausführliche Tabellen, eingefügt werden.

% Ehrenwörtliche Erklärung
\newpage
\addcontentsline{toc}{section}{Ehrenwörtliche Erklärung}
\section*{\texttt{Ehrenwörtliche Erklärung}}
Hiermit erkläre ich, dass ich die vorliegende schriftliche Ausarbeitung im Modul \textbf{Webtechnologies and Applications} selbstständig
angefertigt habe. Es wurden nur die in der Arbeit ausdrücklich benannten Quellen und
Hilfsmittel benutzt. Wörtlich oder sinngemäß übernommenes Gedankengut habe ich als
solches kenntlich gemacht. Diese Arbeit hat in gleicher oder ähnlicher Form noch keiner
Prüfungsbehörde vorgelegen.

\vspace{3cm}
\noindent\begin{tabular}{p{0.5\textwidth}p{0.5\textwidth}}
    \hrulefill & \hrulefill \\
    Ort, Datum & Unterschrift \\
\end{tabular}

\end{document}

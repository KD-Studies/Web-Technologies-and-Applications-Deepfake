\documentclass[a4paper,12pt]{article}

% Pakete für zusätzliche Funktionen
\usepackage[utf8]{inputenc}   % UTF-8 Zeichencodierung
\usepackage[T1]{fontenc}     % Korrekte Silbentrennung und Umlaute
\usepackage[ngerman]{babel}  % Deutsche Spracheinstellungen
\usepackage{amsmath, amssymb} % Mathematische Symbole
\usepackage{graphicx}        % Einfügen von Bildern
\usepackage{hyperref}        % Hyperlinks
\usepackage{geometry}        % Seitenränder
\geometry{a4paper, margin=2.5cm}
\usepackage{listings}        % Code-Listings
\usepackage{csquotes}        % Zitate
\usepackage{xurl}
% Einstellungen für Hyperlinks
\hypersetup{
    colorlinks=true,
    linkcolor=blue,
    citecolor=blue,
    filecolor=magenta,
    urlcolor=cyan,
    pdftitle={Webtechnologies and Applications (WTA)},
    pdfpagemode=FullScreen,
}

% Titel und Autoreninformationen
\title{\textbf{Webtechnologies and Applications (WTA)}}
\author{Kürsat Darcan | MFWS422A}
\date{Abgabedatum: \today}

\begin{document}

% Deckblatt
\maketitle
\thispagestyle{empty}
\vspace{2cm}
\begin{center}
    \includegraphics[width=0.3\textwidth]{FHDW_Logo_RGB-01.svg.png} % Ersetzen Sie "example-image" durch Ihr Logo/Bild
    \\
    \vspace{1cm}
    \textbf{Studiengang: Wirtschaftsinformatik}\\
    \textbf{Fachhochschule der Wirtschaft (FHDW)}
\end{center}
\newpage

% Inhaltsverzeichnis
\renewcommand{\thepage}{\roman{page}} % Seitenzahlen in römischen Ziffern für Verzeichnisse
\tableofcontents
\newpage

% Bei Bedarf Abbildungsverzeichnis
\listoffigures
\addcontentsline{toc}{section}{Abbildungsverzeichnis}
\newpage

% Bei Bedarf Tabellenverzeichnis
\listoftables
\addcontentsline{toc}{section}{Tabellenverzeichnis}
\newpage

% Abkürzungsverzeichnis
\section*{Abkürzungsverzeichnis}
\addcontentsline{toc}{section}{Abkürzungsverzeichnis}
\begin{description}
    \item[HTML] Hypertext Markup Language
    \item[CSS] Cascading Style Sheets
    \item[JS] JavaScript
\end{description}
\newpage

% Hauptteil mit arabischen Seitenzahlen
\renewcommand{\thepage}{\arabic{page}} % Seitenzahlen in arabischen Ziffern für den Hauptteil
\setcounter{page}{1}


% Einleitung
\section{Einleitung}
Mit dem Fortschritt der künstlichen Intelligenz (KI) entwickeln sich auch Deepfake-Technologien stetig weiter, 
wodurch die öffentliche Aufmerksamkeit für diese Thematik zunimmt. 
Diese Technologien ermöglichen die Manipulation digitaler Medieninhalte wie Videos, 
Bilder und Audiodateien, sodass diese täuschend echt wirken, 
obwohl sie nie in dieser Form erstellt wurden. 
Deepfakes sind jedoch keine eigenständige Technologie, 
sondern basieren auf einer Kombination verschiedener KI-Methoden. \cite{BVDW2024}\\
Die vielseitigen Einsatzmöglichkeiten von Deepfakes reichen von kreativen Anwendungsbereichen, 
wie der Erstellung von Unterhaltung, bis hin zu manipulativen oder schädlichen Absichten. 
So können Deepfakes gezielt eingesetzt werden, um die öffentliche Meinung zu beeinflussen, 
Personen oder Unternehmen zu schädigen oder Desinformation zu verbreiten. \cite{CounterExtremism2020}\\
Diese Ausarbeitung untersucht mögliche Gegenmaßnahmen zur Erkennung und Bekämpfung von Deepfake-Technologien. 
Dabei wird auch die ethische Seite betrachtet, 
insbesondere die Frage, welche Verantwortung Entwickler und Nutzer dieser Technologie tragen. 
Abschließend wird ein Ausblick auf die zukünftige Entwicklung der Technologie und deren mögliche Auswirkungen auf die Gesellschaft gegeben.
\newpage

% Gegenmaßnahmen gegen Deepfakes
\section{Gegenmaßnahmen gegen Deepfakes}
Die Verbreitung von Deepfakes stellt eine Herausforderung da,
da diese für Kriminelle zwecke verwendet werdern kann und 
somit auf unterschiedlichen ebenen also auf technischer, gesellschaftlicher als auch rechtlicher Ebene abgesichert werden muss.\cite{CounterExtremism2020}

\subsection{Technologische Lösungen}

\subsection{Gesetzliche Maßnahmen}
\subsection{Prävention und Bildung}
\subsubsection{Öffentliche Aufklärung}
\subsubsection{Tools für Nutzer}
\newpage

% Ethik und Deepfakes
\section{moralische Herausforderungen}
\subsection{Verantwortung der Entwickler}
\subsubsection{Innovation vs. Missbrauch}
\subsubsection{Regulierung von Open-Source-Projekten}
\subsection{Gesellschaftliche Folgen}
\subsubsection{Manipulation der öffentlichen Meinung}
\subsubsection{Persönliche Schäden}
\newpage

% Zukunftsperspektiven
\section{Zukunftsperspektiven}s
\subsection{Technologische Entwicklungen}
\subsection{Gesellschaftliche Anpassungen}
\newpage

%Fazit
\section{Fazit}



% Literaturverzeichnis
\newpage
\addcontentsline{toc}{section}{Literaturverzeichnis}
\section*{Literaturverzeichnis}
\begin{thebibliography}{99}
    \bibitem{Beispiel1} Beispielautor, B. (2025). 
    \textit{Titel des Buches}. Verlag, Ort.
    \bibitem{Beispiel2} Beispielautor, C. und Beispielautor, D. (2024). Artikelname. 
    \textit{Zeitschrift}, 12(3), 45-67.
    \bibitem{BSI2025} Bundesamt für Sicherheit in der Informationstechnik (BSI). (2025). Deepfakes – Bedrohungspotenziale und Gegenmaßnahmen. 
    Verfügbar unter: \url{https://www.bsi.bund.de/DE/Themen/Unternehmen-und-Organisationen/Informationen-und-Empfehlungen/Kuenstliche-Intelligenz/Deepfakes/deepfakes_node.html} (zuletzt aufgerufen am 21.01.2025) Kopie aufbewahrt als \texttt{BSI - Deepfakes - Gefahren und Gegenmaßnahmen.pdf}.
    \bibitem{BVDW2024} Bundesverband Digitale Wirtschaft (BVDW). (2024). Deepfakes – Eine juristische Einordnung. 
    Verfügbar unter: \url{https://www.bvdw.org/wp-content/uploads/2024/07/2024_BVDW_Deepfakes.pdf} (zuletzt aufgerufen am 22.01.2025). Kopie aufbewahrt als \texttt{2024\_BVDW\_Deepfakes\_eine\_einordnung.pdf}.
    \bibitem{CounterExtremism2020} Counter Extremism Project. (2020). Deepfakes – Bedrohungspotenziale und Gegenmaßnahmen.
    Verfügbar unter: \url{https://www.counterextremism.com/sites/default/files/200806_Deep_Fakes_DE_WEB_DS.pdf} (zuletzt aufgerufen am 22.01.2025). Kopie aufbewahrt als \texttt{200806\_Deep\_Fakes\_DE\_WEB\_DS.pdf}.
    \bibitem{Kaur2024} Kaur, A., Noori Hoshyar, A., Saikrishna, V., Firmin, S., \& Xia, F. (2024). Deepfake video detection: challenges and opportunities. Verfügbar unter: 
    \url{https://link.springer.com/content/pdf/10.1007/s10462-024-10810-6.pdf} (zuletzt aufgerufen am 23.01.2025).
    Kopie aufbewahrt als \texttt{Deepfake\_video\_detection\_challenges\_and\_opportunities.pdf}.


\end{thebibliography}

\newpage
% Anhang
\appendix
\section{Anhang}
Hier können zusätzliche Informationen, wie Code-Beispiele oder ausführliche Tabellen, eingefügt werden.

% Ehrenwörtliche Erklärung
\newpage
\addcontentsline{toc}{section}{Ehrenwörtliche Erklärung}
\section*{\texttt{Ehrenwörtliche Erklärung}}
Hiermit erkläre ich, dass ich die vorliegende schriftliche Ausarbeitung im Modul \textbf{Webtechnologies and Applications} selbstständig
angefertigt habe. Es wurden nur die in der Arbeit ausdrücklich benannten Quellen und
Hilfsmittel benutzt. Wörtlich oder sinngemäß übernommenes Gedankengut habe ich als
solches kenntlich gemacht. Diese Arbeit hat in gleicher oder ähnlicher Form noch keiner
Prüfungsbehörde vorgelegen.

\vspace{3cm}
\noindent\begin{tabular}{p{0.5\textwidth}p{0.5\textwidth}}
    \hrulefill & \hrulefill \\
    Ort, Datum & Unterschrift \\
\end{tabular}

\end{document}

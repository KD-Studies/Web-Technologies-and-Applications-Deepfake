\documentclass[a4paper,12pt]{article}

% Pakete für zusätzliche Funktionen
\usepackage[utf8]{inputenc}   % UTF-8 Zeichencodierung
\usepackage[T1]{fontenc}     % Korrekte Silbentrennung und Umlaute
\usepackage[ngerman]{babel}  % Deutsche Spracheinstellungen
\usepackage{amsmath, amssymb} % Mathematische Symbole
\usepackage{graphicx}        % Einfügen von Bildern
\usepackage{hyperref}        % Hyperlinks
\usepackage{geometry}        % Seitenränder
\geometry{a4paper, margin=2.5cm}
\usepackage{listings}        % Code-Listings
\usepackage{csquotes}        % Zitate
\usepackage{xurl}
% Einstellungen für Hyperlinks
\hypersetup{
    colorlinks=true,
    linkcolor=blue,
    citecolor=blue,
    filecolor=magenta,
    urlcolor=cyan,
    pdftitle={Webtechnologies and Applications (WTA)},
    pdfpagemode=FullScreen,
}

% Titel und Autoreninformationen
\title{\textbf{Webtechnologies and Applications (WTA)}}
\author{Kürsat Darcan | MFWS422A}
\date{Abgabedatum: \today}

\begin{document}
% Deckblatt
\maketitle
\thispagestyle{empty}
\vspace{2cm}
\begin{center}
    \includegraphics[width=0.3\textwidth]{FHDW_Logo_RGB-01.svg.png} % Ersetzen Sie "example-image" durch Ihr Logo/Bild
    \\
    \vspace{1cm}
    \textbf{Studiengang: Wirtschaftsinformatik}\\
    \textbf{Fachhochschule der Wirtschaft (FHDW)}
\end{center}
\newpage

% Inhaltsverzeichnis
\renewcommand{\thepage}{\roman{page}} % Seitenzahlen in römischen Ziffern für Verzeichnisse
\tableofcontents
\newpage

% Bei Bedarf Abbildungsverzeichnis
%\listoffigures
%\addcontentsline{toc}{section}{Abbildungsverzeichnis}
%\newpage

% Bei Bedarf Tabellenverzeichnis
%\listoftables
%\addcontentsline{toc}{section}{Tabellenverzeichnis}
%\newpage

% Abkürzungsverzeichnis
\section*{Abkürzungsverzeichnis}
\addcontentsline{toc}{section}{Abkürzungsverzeichnis}
\begin{description}
    \item[CCN] Hypertext Markup Language
    \item[KI] Hypertext Markup Language
    \item[GAN] Hypertext Markup Language
    \item[C2PA]
\end{description}
\newpage

% Hauptteil mit arabischen Seitenzahlen
\renewcommand{\thepage}{\arabic{page}} % Seitenzahlen in arabischen Ziffern für den Hauptteil
\setcounter{page}{1}


% Einleitung
\section{Einleitung}
\subsection{Hintergrund und Relevanz des Themas}
Mit dem Fortschritt der künstlichen Intelligenz (KI) entwickeln sich auch Deepfake-Technologien stetig weiter, 
wodurch die öffentliche Aufmerksamkeit für diese Thematik zunimmt. 
Diese Technologien ermöglichen die Manipulation digitaler Medieninhalte wie Videos, 
Bilder und Audiodateien, sodass diese täuschend echt wirken, 
obwohl sie nie in dieser Form erstellt wurden. 
Deepfakes sind jedoch keine eigenständige Technologie, 
sondern basieren auf einer Kombination verschiedener KI-Methoden. \cite{BVDW2024}\\

\subsection{Problemstellung und Herausforderung}
Die vielseitigen Einsatzmöglichkeiten von Deepfakes reichen von kreativen Anwendungsbereichen, 
wie der Erstellung von Unterhaltung, bis hin zu manipulativen oder schädlichen Absichten. 
So können Deepfakes gezielt eingesetzt werden, um die öffentliche Meinung zu beeinflussen, 
Personen oder Unternehmen zu schädigen oder Desinformation zu verbreiten. \cite{CounterExtremism2020}\\

\subsection{Ziel der Ausarbeitung und Aufbau}
Diese Ausarbeitung untersucht mögliche Gegenmaßnahmen zur Erkennung und Bekämpfung von Deepfake-Technologien. 
Dabei wird auch die ethische Seite betrachtet, 
insbesondere die Frage, welche Verantwortung Entwickler und Nutzer dieser Technologie tragen. 
Abschließend wird ein Ausblick auf die zukünftige Entwicklung der Technologie und deren mögliche Auswirkungen auf die Gesellschaft gegeben.
\newpage

% Gegenmaßnahmen gegen Deepfakes
\section{Gegenmaßnahmen gegen Deepfakes}
Bevor auf die Gegenmaßnahmen eingegangen wird,
ist es wichtig zu verstehen, wie genau Deepfakes erstellt beziehungsweise generiert werden.
Durch das Voranschreiten der künstlichen Intelligenz und im maschinellen Lernen entstand
das Generative Adversarial Networks(GANs) durch Ian Goodfellow im Jahre 2014.
Die GAN besteht aus zwei neuronalen Netzwerken, die gegeneinander trainieren.
Diese beiden Netzwerke werden Generator und Diskriminator bezeichnet.\\
Beim Generator werden Deepfakes erstellt beziehungsweise neue Images erzeugt,
die so weit wie möglich realistisch aussehen.\\ 
Im Diskriminator werden die Trainingsdaten mit den generierten Images vermischt.
Sodass der Diskriminator dann entscheidet, ob es ein Original oder generiertes (Fake) Image ist.\\
Durch diese Zusammenarbeit versuchen sich die Netzwerke zu verbessern, indem der Generator versucht, realistische Images zu erstellen und der Diskriminator die generierten Images zu erkennen.\cite{BVDW2024}

\subsection{KI-Erkennungsmechanismen}\label{subsec:KI-Erkennungsmechanismen}
Nachdem die Funktionsweise von Deepfakes erklärt wurde, sollte klar sein,
dass die Erkennung von Deepfakes sich kompliziert gestaltet.
Auch bei den Erkennungsmechanismen werden unterschiedliche Technologien, algorithmische und menschliche Ansätze kombiniert.
Um Deepfakes zu erkennen, werden aktuell Methoden verwendet, die nach spezifischen Mustern in der Bild- oder Tonverarbeitung suchen.
Es werden Unregelmäßigkeiten in den Bildern oder Videos gesucht, die für das menschliche Auge schwer,
aber für den Algorithmus leicht zu erkennen sind.\\
Ein weiterer Ansatz ist die Verwendung von Convolutional Neural Networks (CNNs)
die auf subtile Anomalien trainiert werden, wie auf Gesichtsbewegungen oder Sprachmodulationen.
Auch werden die Blinzelmuster analysiert, die bei Deepfakes oftmals fehlen oder unnatürlich erscheinen.\cite{BVDW2024}

\subsection{Deklarierung von KI-generierten Medien}
Durch die Generierung von KI-generierten Medien werden viele Inhalte auch in sozialen Netzwerken wie Instagram, 
YouTube und mehr verbreitet. 
Da die Deepfakes realistisch gestaltet werden, 
ist es für den Konsumenten schwer zu unterscheiden, 
ob die kommenden Reels echt oder generiert sind.
Auf der Social-Media-Plattform YouTube wird der C2PA-Standard (Coalition for Content Provenance and Authenticity) verwendet, 
um beim Upload eines Videos die Metadaten zu extrahieren, 
die folgenden Informationen enthalten: Kamera-Modell, 
Zeitpunkt der Aufnahme, Software-Verlauf und Signaturen. 
Um zu prüfen, ob das hochgeladene Video mittels KI geändert wurde, 
wird die Authentizität der Metadaten anhand der Signaturen geprüft. 
Des Weiteren wird geprüft, ob das Video C2PA unterstützt und ob die kryptografische Signatur in der Datei enthalten ist.
Jedes Video wird in kleine Datenblöcke unterteilt, 
die jeweils einen Hash-Wert, also eine Prüfziffer, 
enthalten. Diese wird dann mit der Original-Prüfziffer verglichen. 
Falls diese nicht übereinstimmen, wird davon ausgegangen, 
dass das Video durch KI angepasst wurde.\cite{TheVerge2024a}
Währenddessen setzt Meta, der Publisher von Instagram, Facebook etc., auf externe Unternehmen wie Adobe oder OpenAI,
die bei KI-generierten Inhalten Wasserzeichen oder in den Metadaten Informationen hinterlegen,
um zu erkennen, ob ein Inhalt KI-generiert ist oder nicht. Falls beides,
also das Wasserzeichen oder die nötige Information in den Metadaten,
nicht vorhanden ist, verwendet Meta die Methode der KI-Erkennungsmechanismen in Form der CCNs (wie im Kapitel \ref{subsec: KI-Erkennungsmechanismen} beschrieben wurde).\cite{TheVerge2024b}\cite{Meta2024}

\subsection{Prävention und Bildung}
Prävention und Bildung spielen eine zentrale Rolle bei der Bekämpfung von Deepfakes.
Durch die ernsthafte Gefahr, die durch die manipulierten Medieninhalte hervortreten kann,
wie in der öffentlichen Meinungsbildung, das Vertrauen in digitale Medien oder politische Integrität.
Dadurch ist es notwendig, das Bewusstsein zu schärfen und die Erkennung von Deepfakes zu fördern.
Diese können durch gezielte öffentliche Aufklärung und den Ausbau von Medienkompetenz sowie die Sensibilisierung für digitale Medien durch öffentliche Einrichtungen oder die Bundesregierung 
(vgl. Bundesregierung \cite{Bundesregierung2024}, vgl. Bundeszentrale für politische Bildung \cite{BPB2024}, vgl. Medienkompetenz NRW \cite{Medienkompetenz2024},
vgl. Bundesministerium für Bildung, Wissenschaft und Forschung \cite{Erwachsenenbildung2024})

\newpage

% Ethik und Deepfakes
\section{Moralische Herausforderungen}
Nachdem die Gegenmaßnahmen erläutert wurden, mit denen sich eine Person gegen Deepfakes schützen kann, müssen nun die moralischen Aspekte dieser Technologie genauer betrachtet werden.
Zuvor ist es jedoch essenziell, den Begriff Moral klar zu definieren und ihre Grenzen zu bestimmen, um in den folgenden Kapiteln bestimmen zu können, welches Handeln von den jeweiligen Beteiligten  moralisch akzeptiert werden kann oder was moralisch verachtet werden soll.

\subsection{Definition von Moral}
Immanuel Kant war ein bedeutender Philosoph, der den kategorischen Imperativ formulierte:
\textit{"Handle nur nach derjenigen Maxime, durch die du zugleich wollen kannst, dass sie ein allgemeines Gesetz werde."} \cite{KantMetaphysik}.

Dieses Zitat hebt hervor, wo die Grenzen der Moral gesetzt werden. Jede Person soll ihr Handeln reflektieren und dementsprechend agieren.
Ein Negativbeispiel hierfür wäre ein falsches Versprechen. Eine Person, die Geld leihen möchte, aber weiß, dass sie dieses Geld nicht zurückzahlen kann, muss sich laut Kants Prinzip die Frage stellen, ob die Maxime der Person ein allgemeines Gesetz werden könnte. Die Folgen dazu wären, dass, wenn jede Person dieses Vertrauen missbrauchen würde, niemand jemals wieder jemandem vertrauen würde. Daher würde die falsche Versprechung moralisch falsch betrachtet werden. \cite{KantMetaphysik}
Diese Definition von Moral wird für die weitere Analyse in Betracht gezogen, da diese von Kant als universelle Prinzipien basiert und für alle Menschen gleichermaßen gelten sollte.

\subsection{Verantwortung über Deepfakes}
Deepfakes bringen große ethische und rechtliche Herausforderungen mit sich. Wer ist eigentlich dafür verantwortlich, wenn sie erstellt, verbreitet oder für schädliche Zwecke genutzt werden?

Nach Kants kategorischem Imperativ sollte sich jeder, der Deepfakes produziert oder teilt, die Frage stellen: Was wäre, wenn das jeder tun würde? Würde es in Ordnung sein, wenn manipulierte Inhalte alltäglich wären und möglicherweise Schaden anrichten? Wahrscheinlich nicht, denn das würde das Vertrauen in digitale Medien zerstören und ernste gesellschaftliche Folgen haben.\cite{KantMetaphysik}

Daher gibt es verschiedene Akteure, die für den Umgang mit Deepfakes verantwortlich sind.

In den folgenden Übersicht wird nur die Ehetische Imperativ von Kant betrachtet und nicht im Tiefen besprochen, da diese den Rahmen der Ausarbeitung sprengen würde.

Ersteller von Deepfakes:
Diejenigen die Deepfakes erstellen, müssen sich auf bestimmte Moralische Aspekt im klaren sein, wie genau dieser Deepfake verwendet werden soll. Das heißt wenn ein Deepfake verwendet wird um Täuschung und Manipulation abzuzielen, muss  dieser aus dem sicht von Kant die folgende frage stellen "Welche Auswirkung, wenn jeder Deepfakes verbreitet mit dem Zweck von Täuschung und Manipulation zu erlangen". Durch diese Fragestellung kommt hervor, dass Vertrauen in Informationen und Medien kein glauben mehr bekommen würden, wenn das gezielt von jeden verbreitet werden würde. Auch wenn ein Deepfake für Unterhaltung erstellt wird muss beachtet werden, ob dieses Deepfake "Missbraucht" werden könnte, was somit wieder die Fragestellung von Kant mit sich bringt, "welche folgen hat ein Deepfake, wenn dieser Missbraucht wird".\cite{IJS2025}\cite{KantMetaphysik}

Deepfakes Plattformanbieter:
Auch Plattformanbieter müssen darauf acht geben wie  genau, sie wollen, dass Ihr Plattform behandelt verwendet werden soll, in Zusammenhang von Kant sollten auch die Betreiber sich die Fragestellen "Wie kann/werden die generierten Deepfakes verwendet?" In Zuge der Fragestellung werden auch mögliche Maßnahmen mit betrachtet die diese Moralische Frage vielseitig : wie Zum Beispiel Transparenz oder Kennzeichnungspflicht von Deepfakes, auch die Implementierung von Sicherheitsmaßnahmen von erstellten Deepfakes, ob ein Deepfake als schädlich eingestuft wird oder nichts
Aus dem zusammenhing könnte die Moralische sicht weise vom eingeschränkt werden.
\cite{Brennan2024}\cite{KantMetaphysik}

Gesetzgeber:
Auch die Regierung beziehungsweise die Gesetzgeber spielen eine zentraler Rolle  in der heutigen Gesellschaft. auch sollten die Gesetze menschenwürdig sein, so das die Bürger nicht beeinträchtigt werden durch Deepfakes.\cite{ThomsonReuters2024}\cite{KantMetaphysik}

Konsumenten:
Durch Kants Imperativ wird hervorgehoben, dass die Konsumenten nicht blindlings alles glauben sollten 
und auch keine weitere Beeinflussung bei anderen Bürger zu verursachen.
Auch sollten die Konsumente aktiv daran beteiligen, sich gegen solche Deepfakes zu widmen und aktiv der Aufklärung zu beteiligen.\cite{UNR2023}\cite{KantMetaphysik}
\newpage

% Zukunftsperspektiven
\section{Zukunftsperspektiven}
\subsection{Technologische Entwicklungen}
\subsection{Gesellschaftliche Anpassungen}
\newpage

%Fazit
\section{Fazit}



% Literaturverzeichnis
\newpage
\addcontentsline{toc}{section}{Literaturverzeichnis}
\section*{Literaturverzeichnis}
\begin{thebibliography}{99}
    \bibitem{Arxiv2020} Nguyen, H. H., Yamagishi, J., \& Echizen, I. (2020). Capsule-Forensics: Using Capsule Networks to Detect Forged Images and Videos.
    Verfügbar unter: \url{https://arxiv.org/pdf/2003.08685} (zuletzt aufgerufen am 23.01.2025).
    
    \bibitem{Basecamp2024} Basecamp Digital. (2024). KI verstehen: Update zu Kennzeichnungspflichten und zum neuen ChatGPT. 
    Verfügbar unter: \url{https://www.basecamp.digital/ki-verstehen-update-zu-kennzeichnungspflichten-und-zum-neuen-chatgpt/} (zuletzt aufgerufen am 23.01.2025).
    
    \bibitem{BPB2024} Bundeszentrale für politische Bildung (BPB). (2024). Regulierung von Deepfakes. 
    Verfügbar unter: \url{https://www.bpb.de/lernen/bewegtbild-und-politische-bildung/556822/regulierung-von-deepfakes/} (zuletzt aufgerufen am 23.01.2025). 
    
    \bibitem{BPBDeepfake2024} Bundeszentrale für politische Bildung (BPB). (2024). Technische Ansätze zur Deepfake-Erkennung und -Prävention. 
    Verfügbar unter: \url{https://www.bpb.de/lernen/bewegtbild-und-politische-bildung/556855/technische-ansaetze-zur-deepfake-erkennung-und-praevention/} (zuletzt aufgerufen am 23.01.2025).
    
    \bibitem{BSI2025} Bundesamt für Sicherheit in der Informationstechnik (BSI). (2025). Deepfakes \u2013 Bedrohungspotenziale und Gegenmaßnahmen. 
    Verfügbar unter: \url{https://www.bsi.bund.de/DE/Themen/Unternehmen-und-Organisationen/Informationen-und-Empfehlungen/Kuenstliche-Intelligenz/Deepfakes/deepfakes_node.html} (zuletzt aufgerufen am 21.01.2025).
    
    \bibitem{BVDW2024} Bundesverband Digitale Wirtschaft (BVDW). (2024). Deepfakes \u2013 Eine juristische Einordnung. 
    Verfügbar unter: \url{https://www.bvdw.org/wp-content/uploads/2024/07/2024_BVDW_Deepfakes.pdf} (zuletzt aufgerufen am 22.01.2025). 
    
    \bibitem{Bundesdruckerei2024} Bundesdruckerei. (2024). Desinformation und Fake-ID. 
    Verfügbar unter: \url{https://www.bundesdruckerei.de/de/innovation-hub/desinformation-und-fake-id} (zuletzt aufgerufen am 23.01.2025). 
    
    \bibitem{CounterExtremism2020} Counter Extremism Project. (2020). Deepfakes \u2013 Bedrohungspotenziale und Gegenmaßnahmen.
    Verfügbar unter: \url{https://www.counterextremism.com/sites/default/files/200806_Deep_Fakes_DE_WEB_DS.pdf} (zuletzt aufgerufen am 22.01.2025).
    
    \bibitem{Fraunhofer2024} Fraunhofer ISI. (2024). Chancen und Risiken von Deepfakes für Politik, Wirtschaft und Gesellschaft. 
    Verfügbar unter: \url{https://www.isi.fraunhofer.de/de/presse/2024/presseinfo-17-deepfakes-chancen-risiken-Politik-Wirtschaft-Gesellschaft.html} (zuletzt aufgerufen am 23.01.2025). 
    
    \bibitem{Kaur2024} Kaur, A., Noori Hoshyar, A., Saikrishna, V., Firmin, S., \& Xia, F. (2024). Deepfake video detection: challenges and opportunities. 
    Verfügbar unter: \url{https://link.springer.com/content/pdf/10.1007/s10462-024-10810-6.pdf} (zuletzt aufgerufen am 23.01.2025).
    
    \bibitem{Meta2024} Meta. (2024). How AI-generated content is identified and labeled on Meta. 
    Verfügbar unter: \url{https://www.meta.com/de-de/help/artificial-intelligence/how-ai-generated-content-is-identified-and-labeled-on-meta/} (zuletzt aufgerufen am 23.01.2025).
    
    \bibitem{NRW2024} Medienkompetenzrahmen NRW. (2024). Desinformation und Deepfakes mit Medienkompetenz begegnen. 
    Verfügbar unter: \url{https://medienkompetenzrahmen.nrw/aktuelles/detail/desinformation-und-deepfakes-mit-medienkompetenz-begegnen} (zuletzt aufgerufen am 23.01.2025).
    
    \bibitem{Springer2024} Dittmann, J., \& Vielhauer, C. (2024). Technische Ansätze zur Deepfake-Erkennung. In: Künstliche Intelligenz und digitale Ethik. Springer. 
    Verfügbar unter: \url{https://link.springer.com/chapter/10.1007/978-3-662-65964-9_10} (zuletzt aufgerufen am 23.01.2025).
    
    \bibitem{TheVerge2024a} The Verge. (2024). YouTube introduces C2PA standard for captured camera labels and content credentials. 
    Verfügbar unter: \url{https://www.theverge.com/2024/10/15/24271083/youtube-c2pa-captured-camera-label-content-credentials} (zuletzt aufgerufen am 23.01.2025).
    
    \bibitem{TheVerge2024b} The Verge. (2024). Facebook and Instagram update AI label for edited content. 
    Verfügbar unter: \url{https://www.theverge.com/2024/9/12/24242998/facebook-instagram-ai-label-update-edited-content} (zuletzt aufgerufen am 23.01.2025).
    
    \bibitem{Tibor2024} Tibor.net. (2024). Instagram markiert Posts mit \u201eKI-generiert\u201c \u2013 was bedeutet das? 
    Verfügbar unter: \url{https://tibor.net/digital/kuenstliche-intelligenz/instagram-markiert-posts-mit-mit-ki-generiert-was-bedeutet-das/} (zuletzt aufgerufen am 23.01.2025).
    
    \bibitem{Bundesregierung2024} Bundesregierung. (2024). Deepfakes. 
    Verfügbar unter: \url{https://www.bundesregierung.de/breg-de/aktuelles/deepfakes-2246064} (zuletzt aufgerufen am 03.02.2025).
    
    \bibitem{BPB2024b} Bundeszentrale für politische Bildung (BPB). (2024). Deepfakes – Wenn man Augen und Ohren nicht mehr trauen kann. 
    Verfügbar unter: \url{https://www.bpb.de/lernen/digitale-bildung/werkstatt/542670/deepfakes-wenn-man-augen-und-ohren-nicht-mehr-trauen-kann/} (zuletzt aufgerufen am 03.02.2025).
    
    \bibitem{Erwachsenenbildung2024} Erwachsenenbildung.at. (2024). Deepfakes und Erwachsenenbildung. 
    Verfügbar unter: \url{https://erwachsenenbildung.at/digiprof/neuigkeiten/19562-deepfakes-und-erwachsenenbildung.php} (zuletzt aufgerufen am 03.02.2025).
    
    \bibitem{Medienkompetenz2024} Medienkompetenzrahmen NRW. (2024). Desinformation und Deepfakes – Mit Medienkompetenz begegnen. 
    Verfügbar unter: \url{https://medienkompetenzrahmen.nrw/aktuelles/detail/desinformation-und-deepfakes-mit-medienkompetenz-begegnen} (zuletzt aufgerufen am 03.02.2025).
    
    \bibitem{KantMetaphysik} Kant, I. (1785). Grundlegung zur Metaphysik der Sitten.  
    Verfügbar unter: \url{http://www.zeno.org/Philosophie/M/Kant,+Immanuel/Grundlegung+zur+Metaphysik+der+Sitten/Zweiter+Abschnitt%3A+%C3%9Cbergang+von+der+popul%C3%A4ren+sittlichen+Weltweisheit+zur+Metaphysik+der+Sitten} (zuletzt aufgerufen am 05.02.2025).  
    \bibitem{IJS2025} Internet Just Society. (2025). Legal Issues of Deepfakes.
    Verfügbar unter: \url{https://www.internetjustsociety.org/legal-issues-of-deepfakes} (zuletzt aufgerufen am 10.02.2025).

    \bibitem{Brennan2024} Brennan Center for Justice. (2024). Regulating AI, Deepfakes, and Synthetic Media in the Political Arena.  
    Verfügbar unter: \url{https://www.brennancenter.org/our-work/research-reports/regulating-ai-deepfakes-and-synthetic-media-political-arena} (zuletzt aufgerufen am 10.02.2025).

    \bibitem{ThomsonReuters2024} Thomson Reuters. (2024). Deepfakes: Federal and State Regulation.  
    Verfügbar unter: \url{https://www.thomsonreuters.com/en-us/posts/government/deepfakes-federal-state-regulation/} (zuletzt aufgerufen am 10.02.2025).

    \bibitem{UNR2023} University of Nevada, Reno. (2023). Deepfakes: Emerging Threats and Challenges.
    Verfügbar unter: \url{https://www.unr.edu/nevada-today/news/2023/atp-deepfakes} (zuletzt aufgerufen am 10.02.2025).

\end{thebibliography}
\newpage
% Anhang
\appendix
\section{Anhang}
Hier können zusätzliche Informationen, wie Code-Beispiele oder ausführliche Tabellen, eingefügt werden.

% Ehrenwörtliche Erklärung
\newpage
\addcontentsline{toc}{section}{Ehrenwörtliche Erklärung}
\section*{\texttt{Ehrenwörtliche Erklärung}}
Hiermit erkläre ich, dass ich die vorliegende schriftliche Ausarbeitung im Modul \textbf{Webtechnologies and Applications} selbstständig
angefertigt habe. Es wurden nur die in der Arbeit ausdrücklich benannten Quellen und
Hilfsmittel benutzt. Wörtlich oder sinngemäß übernommenes Gedankengut habe ich als
solches kenntlich gemacht. Diese Arbeit hat in gleicher oder ähnlicher Form noch keiner
Prüfungsbehörde vorgelegen.

\vspace{3cm}
\noindent\begin{tabular}{p{0.5\textwidth}p{0.5\textwidth}}
    \hrulefill & \hrulefill \\
    Ort, Datum & Unterschrift \\
\end{tabular}

\end{document}
